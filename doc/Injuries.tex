\documentclass[a4paper,9pt]{article}
\usepackage[a4paper]{geometry}
\geometry{top=0.5in, bottom=0.5in, left=0.5in, right=0.5in}
\usepackage{amsmath,amssymb,lmodern} 
\usepackage{cancel}
\usepackage{mathrsfs}
\usepackage{amsfonts}
\usepackage{fullpage}
\usepackage{enumerate}
\usepackage{graphicx}

\title{Probabilidad lesiones.}
\author{Xaunther}

\begin{document}

\maketitle

\section{Introducci\'on}
Aqu\'i se van a explicar las probabilidades de lesi\'on del simulador, matem\'aticamente. Los par\'ametros usados se encuentran en config/Injuries.dat y son los siguientes:
\begin{itemize}
\item Injury: Cada cu\'antos minutos (de media) se producir\'a una lesi\'on en un partido. Por definici\'on, $Injury=1/P_{lesion}$. Si el n\'umero es 180 quiere decir que se producir\'a una lesi\'on cada dos partidos.
\item Injury\_i ($i=0,1,2,3,4,...,N$): Cuando se produce una lesi\'on, cu\'al es la probabilidad de que \'esta conlleve i d\\'ias de baja (en \%). Por tanto, $Injury\_i=P_i/100$ y asumiremos que $\sum_{i=0}^{N}{P_i}=1$
\end{itemize}

\section{Probabilidad de lesi\'on}
La probabilidad de que ocurran k lesiones sigue una distribuci\'on de probabilidad binomial, con $p=P_{lesion}$, y n correspondiente al n\'umero de minutos jugados (m):
\begin{equation}
P(k;m,P_{lesion}) = \binom{m}{k}P_{lesion}^k\left(1-P_{lesion}\right)^{m-k}
\end{equation}
As\'i pues, la media de lesiones en X partidos de 90 minutos($\mu_{lesion}$) es:
\begin{align}
\mu_{lesion}&=mP_{lesion}=90XP_{lesion}=\frac{90X}{Injury}\\
\frac{\mu_{lesion}}{X}&=\frac{90}{Injury}
\end{align}
Y la desviaci\'on est\'andar (que tomaremos como error) es:
\begin{align}
\sigma_{lesion}&=\sqrt{90XP_{lesion}(1-P_{lesion})}\\
\frac{\sigma_{lesion}}{X}&=\sqrt{\frac{90}{X}P_{lesion}(1-P_{lesion})}\\
\frac{\sigma_{lesion}}{X}&=\frac{\sqrt{\frac{90}{X}(Injury-1)}}{Injury}\\
\frac{\sigma_{lesion}}{X}&\approx\sqrt{\frac{90}{X\cdot Injury}}
\end{align}

Donde la aproximaci\'on es v\'alida cuando $Injury >> 1$. Como es esperable, la desviaci\'on est\'andar disminuye con el n\'umero de partidos, y en el caso de $Injury=180$ el n\'umero de lesiones por partido es:
\begin{align}
X=1&\rightarrow 0.5\pm0.7\\
X=2&\rightarrow 0.5\pm0.5\\
X=5&\rightarrow 0.5\pm0.3\\
X=10&\rightarrow 0.5\pm0.2
\end{align}

\section{Probabilidad de duraci\'on de lesi\'on}
La probabilidad de duraci\'on es simplemente la probabilidad de que haya una lesi\'on por la probabilidad de dicha duraci\'on:
\begin{align}
\frac{\mu_{lesion_i}}{X}&=\frac{9}{10}\frac{Injury\_i}{Injury}\\
\frac{\sigma_{lesion}}{X}&\approx\sqrt{\frac{9}{10}\frac{Injury\_i}{X\cdot Injury}}
\end{align}
\end{document}